\documentclass[titlepage, a4paper, 11pt]{scrartcl}

%too much whitespace otherwise
\usepackage[left=35mm,top=26mm,right=26mm,bottom=15mm]{geometry}

% deutsche Übersetzungen
%\usepackage[ngerman]{babel}
% Grafik Pakete
\usepackage{graphicx,hyperref,amssymb}
% Ordner für Grafiken
\graphicspath{ {./images/} }
% Pakete für Formatierung der Grafiken
\usepackage{wrapfig}
\usepackage{float}
% deutsches Encoding (Umlaute)
\usepackage[utf8]{inputenc}
% für Grad Symbol
\usepackage{textcomp}

% Header and Footer
\usepackage{fancyhdr}

%image grid
\usepackage{graphicx}
\usepackage{subfig}

\usepackage{multicol}

\pagestyle{fancy}
\fancyhf{}
\rhead{Lückert, Neudecker}
\lhead{Kalman-filter for partial linear systems for tracking of pool balls}
 
\begin{document}

\title{Kalman-filter for partial linear systems for tracking pool balls}
\author{Marlon Lückert B.Sc. - Julius Neudecker B.Sc.}
\date{März 2020}
\maketitle

\tableofcontents

\begin{abstract}
    In this proposal we are going to discuss an advanced implementation of the Kalman-filter \cite{kalman} to improve measurement quality and predict the movement of balls on a pool table. 
    We are going to point out the necessity of an advanced filter design in this particular case. The Problem we are going to solve is the following:
    Since the performance of the filter deteriorates in cases of a rapid change in direction, the filter has to be able to adapt more quickly to these rapid changes.
    Our plan is to derive two different implementations with adaptive behavior. The implementations will be tested in a simulator and with real world video footage of a pool table.
    The goal is to provide a filter design with a significant lower MSE than a vanilla implementation.
\end{abstract}


\begin{multicols}{2}
    \section{Introduction}
    At first glance the game of pool is very suitable to examine the behavior of a kalman-filter enhanced tracking system based on pure visual tracking. 
    The surface of a pool table is made of a thin fabric which covers a hard surface i.e. slate or granite.
    The balls nowadays are usually made out of resin. This combination of materials creates very small rolling resistance and the balls behave almost fully elastic on collision.
    Since this is only a 2 DOF\footnote{Dimensions Of Freedom - determines the possible rotation or translation along each given axis} problem, this can be solved with a simple linear kalman filter implementation. 
    The problem is, when two balls hit each other or a cushion the velocity vector changes its orientation instantly. 
    If this isn't taken into account, the filter needs some time to adapt to the new direction of movement and will produce wrong estimations during this time.

    This behavior is independent of the type of kalman implementation being CVM or CAM.
    A kalman filter will assume the direction of movement on any given sample is about the same as in the last sample. 
    It will therefore create wrong estimations if the direction of movement changes drastically in a short period of time. The time the filter needs to recover depends on the filter gain.

    We also use the filter to predict values for any given length into the future by feeding back its estimations as actual state. 
    The quality of this prediction however depends on several factors, i.e. the applied process noise and framerate of the video.

    \section{Problem and Motivation}
    

    \section{Hypothesis and Goals}

    \section{State of research}

    \section{Methods of research}

    \section{Proposed timetable}

\end{multicols} 

\bibliography{references} 
\bibliographystyle{ieeetr}


\end{document}