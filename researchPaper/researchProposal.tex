\documentclass[titlepage, a4paper, 11pt]{scrartcl}

%too much whitespace otherwise
\usepackage[left=35mm,top=26mm,right=26mm,bottom=15mm]{geometry}

% deutsche Übersetzungen
%\usepackage[ngerman]{babel}
% Grafik Pakete
\usepackage{graphicx,hyperref,amssymb}
% Ordner für Grafiken
\graphicspath{ {./images/} }
% Pakete für Formatierung der Grafiken
\usepackage{wrapfig}
\usepackage{float}
% deutsches Encoding (Umlaute)
\usepackage[utf8]{inputenc}
% für Grad Symbol
\usepackage{textcomp}

% Header and Footer
\usepackage{fancyhdr}

%image grid
\usepackage{graphicx}
\usepackage{subfig}

\usepackage{multicol}

\pagestyle{fancy}
\fancyhf{}
\rhead{Lückert, Neudecker}
\lhead{Kalman-filter for partial linear systems for tracking of pool balls}
 
\begin{document}

\title{Kalman-filter for partial linear systems for tracking pool balls}
\author{Marlon Lückert B.Sc. - Julius Neudecker B.Sc.}
\date{März 2020}
\maketitle

\tableofcontents

\begin{abstract}
    In this proposal we are going to discuss an advanced implementation of the Kalman-filter \cite{kalman} to improve measurement quality and predict the movement of balls on a pool table. 
    We are going to point out the necessity of an advanced filter design in this particular case. The Problem we are going to solve is the following:
    Since the performance of the filter deteriorates in cases of a rapid change in direction, the filter has to be able to adapt more quickly to these rapid changes.
    Our plan is to derive two different implementations with adaptive behavior. The implementations will be tested in a simulator and with real world video footage of a pool table.
    The goal is to provide a filter design with a significant lower MSE than a vanilla implementation.
\end{abstract}


\begin{multicols}{2}
    \section{Introduction}

    \section{Problem and Motivation}

    \section{Hypothesis and Goals}

    \section{State of research}

    \section{Methods of research}

    \section{Proposed timetable}

\end{multicols} 

\bibliography{references} 
\bibliographystyle{ieeetr}


\end{document}