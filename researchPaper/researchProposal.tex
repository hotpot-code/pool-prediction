\documentclass[titlepage, a4paper, 11pt]{scrartcl}

%too much whitespace otherwise
\usepackage[left=35mm,top=26mm,right=26mm,bottom=15mm]{geometry}

% deutsche Übersetzungen
%\usepackage[ngerman]{babel}
% Grafik Pakete
\usepackage{graphicx,hyperref,amssymb}
% Ordner für Grafiken
\graphicspath{ {./images/} }
% Pakete für Formatierung der Grafiken
\usepackage{wrapfig}
\usepackage{float}
% deutsches Encoding (Umlaute)
\usepackage[utf8]{inputenc}
% für Grad Symbol
\usepackage{textcomp}

% Header and Footer
\usepackage{fancyhdr}

%image grid
\usepackage{graphicx}
\usepackage{subfig}

\usepackage{multicol}

\pagestyle{fancy}
\fancyhf{}
\rhead{Lückert, Neudecker}
\lhead{Kalman-filter for partial linear systems for tracking of pool balls}
 
\begin{document}

\title{Kalman-filter for partial linear systems for tracking pool balls}
\author{Marlon Lückert B.Sc. - Julius Neudecker B.Sc.}
\date{März 2020}
\maketitle

\tableofcontents

\begin{abstract}
    In this proposal we are going to discuss an advanced implementation of the Kalman-filter \cite{kalman} to improve measurement quality and predict the movement of balls on a pool table. 
    We are going to point out the necessity of an advanced filter design in this particular case. The Problem we are going to solve is the following:
    Since the performance of the filter deteriorates in cases of a rapid change in direction, the filter has to be able to adapt more quickly to these rapid changes.
    Our plan is to derive two different implementations with adaptive behavior. The implementations will be tested in a simulator and with real world video footage of a pool table.
    The goal is to provide a filter design with a significant lower MSE than a vanilla implementation.
\end{abstract}


\begin{multicols}{2}

    \section{Introduction}
    At first glance the game of pool is very suitable to examine the behavior of a kalman-filter enhanced tracking system based on pure visual tracking. 
    The surface of a pool table is made of a thin fabric which covers a hard surface i.e. slate or granite.
    The balls nowadays are usually made out of resin. This combination of materials creates very small rolling resistance and the balls behave almost fully elastic on collision.
    Since this is only a 2 DOF\footnote{Dimensions Of Freedom - determines the possible rotation or translation along each given axis} problem, this can be solved with a simple linear kalman filter implementation. 
    The problem is, when two balls hit each other or a cushion the velocity vector changes its orientation instantly. 
    If this isn't taken into account, the filter needs some time to adapt to the new direction of movement and will produce wrong estimations during this time.

    This behavior is independent of the type of kalman implementation being CVM or CAM.
    A kalman filter will assume the direction of movement on any given sample is about the same as in the last sample. 
    It will therefore create wrong estimations if the direction of movement changes drastically in a short period of time. The time the filter needs to recover depends on the filter gain.

    We also use the filter to predict values for any given length into the future by feeding back its estimations as actual state. 
    The quality of this prediction however depends on several factors, i.e. the applied process noise and framerate of the video.

    \section{Problem and Motivation}
    As introduced in the previous section is the lack of adaptability in a vanilla Kalman implementation.
    One could argue that this can be taken into account by using a higher overall process noise which would in turn lead to significantly reduced overall quality.
    This gets worse with increased speed and lower framerate. Essentially rendering the filter useless at a certain point.

    We think this is a good point to develop this algorithm for future application in VR and AR based pool trainers and augmented broadcast experiences.

    \section{Hypothesis and Goals}

    \section{State of research}

    Jong-Yun Kim and Tae-Yong Kim \cite{kim} developed a method to provide robust tracking of a soccer ball. 
    They provide a solution for the problem for the case that the soccer ball might be occluded by the player at any given time,
    which results in a diminished tracking accuracy. 
    In this case they used the velocity vector of the player to substitute for the ball presuming that the ball moves in the same direction as the player does.

    Jia et.al. \cite{jia} conducted research in the trajectory of pool balls, which helped us to decide which kalman model is the most suitable.

    Shiuh et.al. \cite{shiuh} provided a good starting point how to create a tracking algorithm for pool balls. They also developed an algorithm to track occluded objects using an adaptive kalman filter.
    In this case they used to threshold in order to determine whether the object can still be reliably tracked. If this isn't the case the filter will rely only on predicted values until the object can be tracked reliably again.

    Salzmann and Urtasun \cite{salzmann} proposed a more general approach for tracking. 
    They were able to recreate a highly accurate tracking from a noisy picture based on newtons 2nd law and markov models.
    Using different constraints and presumptions they were even able to extract physical parameters like friction and trajectories.

    Mohamed and Schwarz \cite{schwarz} are using partly the same approach as we do to improve the results created by INS/GPS\footnote{Inertial Navigation System / Global Positioning System} Systems.
    However their approach only targets the 'Q' and 'R' parameters of the filter.

    Sarkka and Nummenmaa \cite{sarkka} created an adaptive kalman implementation which adapts itself to time-varying noise parameters. Since our input data is constant in this regard,
    we decided to simulate for the optimal filter parametrization instead of relying on the filter to adapt itself.

    Gabdulkhakova and Kropatsch \cite{kropatsch} use a kalman filter to create a video analysis tool for snooker game broadcasting.

    \section{Methods of research}

    We are going to use two stepped approach: develop and the algorithm in a tailor made simulation
    and evaluate the results with real footage video. Our main instrument of validation will be the MSE
    \footnote{Mean Square Error} between the ground truth and filtered position and the quality of prediction
    with +n frames respectively.


    \section{Proposed timetable}

\end{multicols} 

\bibliography{references} 
\bibliographystyle{ieeetr}


\end{document}