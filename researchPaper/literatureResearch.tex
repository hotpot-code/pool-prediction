\documentclass[titlepage, a4paper, 11pt]{scrartcl}

%too much whitespace otherwise
\usepackage[left=30mm,top=26mm,right=26mm,bottom=15mm]{geometry}

% deutsche Übersetzungen
%\usepackage[ngerman]{babel}
% Grafik Pakete
\usepackage{graphicx,hyperref,amssymb}
% Ordner für Grafiken
\graphicspath{ {./images/} }
% Pakete für Formatierung der Grafiken
\usepackage{wrapfig}
\usepackage{float}
% deutsches Encoding (Umlaute)
\usepackage[utf8]{inputenc}
% für Grad Symbol
\usepackage{textcomp}

% Header and Footer
\usepackage{fancyhdr}

%image grid
\usepackage{graphicx}
\usepackage{subfig}

\usepackage{multicol}

%harvard cite style - use this: \citep{key}
\usepackage{natbib}
\setcitestyle{authoryear,open={(},close={)}}

\pagestyle{fancy}
\fancyhf{}
\rhead{Lückert, Neudecker}
\lhead{Kalman-filter development - literature research}
 
\begin{document}

\section{Research Goal}

Our primary goal with the literature research is to find sources of information which complements our expertise. 

While writing our proposal we identified five aspects where we lack sufficient expertise.
These are in particular:

\paragraph{Statistical experimental design}

As outlined in the research proposal we are going to develop our ideas in a simulation. Therefore it is efficient to design the experimental part in advance.
This means that we are not going to use a trial-and-error approach but instead invest some time to think about possible outcomes and how to achieve these by 
altering different parameters in particular and develop a deeper understanding of their behavior.

\paragraph{Gradient descent algorithms}

Since there exist a huge number of different numerical and analytical ways to solve a gradient descent problem it will save time to evaluate the best
suitable algorithm prior to implementation to avoid unnecessary work.

\textit{This is only for testing the citation format and will be removed later:}
Gradient descent overview \citep{ruder2016overview} is very useful.

\paragraph{Efficient implementation of algorithms}

This is especially important since our algorithm is supposed to work in realtime.
Therefore any delay induced by unefficient code is unacceptable.

\paragraph{openCV best practices}

The analysis of the original footage is made by openCV. Since this library offers a wide variety of different approaches to isolate objecs in videofootage,
it is evaluate the best option.

\paragraph{Measure and comparison of quantifiable data}

The results in our research will be mainly represented by numbers. To create a better understanding of these numbers and how they interact it is a good approach to 
evaluate different methods of putting numbers in context to one another and which scales and graphs are the quasi standard across the research community.


\section{Source of Information}

some lorem text

\section{Criteria for eligible sources}

some lorem text

\section{Sources for our research}

\subsection{Topic A}

happy little lorem

\subsection{Topic B}

happy little lorem

\subsection{Aonther Lorem Topic}

happy little lorem

By the way...somehwere around here are the references to the literature.bib file.

\bibliography{literature} 
\bibliographystyle{abbrvnat}


\end{document}