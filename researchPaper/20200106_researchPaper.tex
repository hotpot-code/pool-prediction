\documentclass[notitlepage, a4paper, 11pt]{scrartcl}

%too much whitespace otherwise
\usepackage[left=35mm,top=26mm,right=26mm,bottom=15mm]{geometry}

% deutsche Übersetzungen
%\usepackage[ngerman]{babel}
% Grafik Pakete
\usepackage{graphicx,hyperref,amssymb}
% Ordner für Grafiken
\graphicspath{ {./images/} }
% Pakete für Formatierung der Grafiken
\usepackage{wrapfig}
\usepackage{float}
% deutsches Encoding (Umlaute)
\usepackage[utf8]{inputenc}
% für Grad Symbol
\usepackage{textcomp}

% Header and Footer
\usepackage{fancyhdr}

%image grid
\usepackage{graphicx}
\usepackage{subfig}

\usepackage{multicol}

\usepackage{cite}

\pagestyle{fancy}
\fancyhf{}
\fancyfoot[C]{\thepage}
\rhead{Lückert, Neudecker}
\lhead{Pool-Predictor - a Kalman-Filter Implementation}
 
\begin{document}

\title{Pool-Predictor - a Kalman-Filter Implementation}
\author{Marlon Lückert B.Sc. - Julius Neudecker B.Sc.}
\date{Feburary 2020}

\maketitle

\begin{abstract}
In this article we're discussing an implementation of a Kalman-Filter to filter and predict the movement of balls on a pool table. 
We discuss the reasoning why we implemented the filter as a constant velocity model and its weak points in this situation.
Since the performance of the filter deteriorates in cases of a rapid change in direction we developed two different implementations with adaptive behaviour.
The implementations were tested in a simulator and with real world video footage of a pool table.
In the end we compared the performaces of each filter to another.

\end{abstract}

\begin{multicols}{2}
\section{Introduction}

\subsection{Game of Pool}

At first glance the game of pool is very suitable to examine the behaviour of a kalman-filter \cite{kalman} enhanced tracking system based on pure visual tracking. 
The surface of a pool table is made of a thin fabric which covers a hard surface i.e. slate or granite.
The balls nowadays are usually made out of resin. This combination of materials creates very small rolling resistance and the balls behave almost fully elastic on collision.
Also with only 2 DOF\footnote{dimensions of freedom} a basic kalman filter implementation is fairly simple in this regard. But when two balls hit each other or a boundary, the velocity vector
changes its orientation instantly. If this isn't taken into account, the filter needs some time to adapt to the new direction of movement and will produce wrong estimations during this time.


\subsection{Visual Recognition of moving elements}

\subsection{Kalman-Filter}
\paragraph{CVM}
\paragraph{CAM}


\section{Related Work}

R E Kalman: https://asmedigitalcollection.asme.org/fluidsengineering/article-abstract/82/1/35/397706

Soccer Ball Tracking with prediction (velocity vector of player to presume ball velocity vector)
Jong-Yun Kim, et al https://ieeexplore.ieee.org/abstract/document/5298809

Trajectory of a Billiard Ball and Recovery of Its Initial Velocities
Yan-Bin Jia, et al http://web.cs.iastate.edu/~jia/papers/billiard-analysis.pdf

Visual Tracking with kalman filter
Shiuh-KuWeng et al https://www.sciencedirect.com/science/article/pii/S1047320306000113

Tracking and Prediction using Markov-Models
Mathieu Salzmann ; Raquel Urtasun https://ieeexplore.ieee.org/abstract/document/6126480

Adaptive Kalman Filtering for INS/GPS (uses filterbanks and innovation based but for very confined situation)
A. H. MohamedK. P. Schwarz https://link.springer.com/article/10.1007/s001900050236

On the identification of variances and adaptive Kalman filtering (our Approach to dynamic filtering  but only steady state)
R. Mehra https://ieeexplore.ieee.org/abstract/document/1099422

Recursive Noise Adaptive Kalman Filtering by Variational Bayesian Approximations
Simo Sarkka ; Aapo Nummenmaa https://ieeexplore.ieee.org/abstract/document/4796261

\section{Implementation}
\subsection{differences between implementations}

\subsection{CVM devel + Problems in Prediction}
\subsection{Dyn CVM devel}
\subsection{Smart Devel}

\section{Simulations}

\subsection{Structure of Simulation}
optimal parameters

\subsection{Sampling}
\subsection{Noise}
\subsection{Velocity}

\section{Results}

\subsection{Real Video - Prediction vs. Filtered}
\paragraph{No Ground Truth}
\subsection{CVM vs. CAM}

\section{Future Work}

\section{Improvements}
Abstand von Bande durch diskrete abtastung ungenau, weil zeitliche unschärfe entsteht (abprall zwischen zwei frames).
Prediction müsste unabhängig von frames gemacht werden.


\subsection{AR App for Training}
\subsection{Broadcasting}

\section{Acknowledgement}

T.Edeler wegen Vorlesung diesdas.

\end{multicols}

\bibliography{references} 
\bibliographystyle{ieeetr}

\end{document}