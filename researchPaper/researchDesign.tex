\documentclass[titlepage, a4paper, 11pt]{scrartcl}

%too much whitespace otherwise
\usepackage[left=30mm,top=26mm,right=26mm,bottom=15mm]{geometry}

% deutsche Übersetzungen
%\usepackage[ngerman]{babel}
% Grafik Pakete
\usepackage{graphicx,hyperref,amssymb}
% Ordner für Grafiken
\graphicspath{ {./images/} }
% Pakete für Formatierung der Grafiken
\usepackage{wrapfig}
\usepackage{float}
% deutsches Encoding (Umlaute)
\usepackage[utf8]{inputenc}
% für Grad Symbol
\usepackage{textcomp}

% Header and Footer
\usepackage{fancyhdr}

%image grid
\usepackage{graphicx}
\usepackage{subfig}

\usepackage{multicol}

%harvard cite style - use this: \citep{key}
\usepackage{natbib}
\setcitestyle{authoryear,open={(},close={)}}

\pagestyle{fancy}
\fancyhf{}
\rhead{Lückert, Neudecker}
\lhead{Kalman-filter development - research design}
 
\begin{document}

\section{Hypothesis}

The tracking and prediction of billiard balls by a Kalman-Filter improves by explicitly taking the physical conditions of a billiards table into account when implementing the filter.

\section{Operationalization}

In the following section we are going to operationalize the different terms in our hypthesis.

\subsection{Tracking}

The tracking describes the deviation from the filtered position of the billiard ball (result of Kalman-Filter) to the real position of the billiard ball.
The position divides into three dimensions, the x-coordinate, the y-coordinate, the z-coordinate.
Since we are investigating a billiard game we disregard the up-direction (z-coordinate), because the ball moves on a flat surface and will not leave the plane.
We are measuring the coordinates in pixels. The range of the pixels is defined by the resolution of the input video on which the Kalman-Filter gets applied to.
The origin, coordinate (0,0), is always the top-left corner.
The final deviation from the two position is calculated with the Euclidean distance, which results in a length in pixels.

The position of the billiard ball changes over time, because we are investigating video sources.
To measure the deviation of the whole video and to get a key figure, we are calculating the mean square error (mse) in which the error is the pixel distance between the two positions. 

\subsection{Prediction}

The prediction describes the deviation from the calculated future position of the billiard ball to the real position at this specific time.
The positions use the same measurements as described in the previous chapter.
The time is measured in frames. The specific number of frames we will look in the future will vary in our experiments.
To make the frame count comparable with different input video source we align them with the framerate (frames per second) of the video.

\subsection{Improvement}

The improvement compares a standard Kalman-Filter to our implementation of the Kalman-Filter.
The improvement is measured as the difference between the different mean square errors of the tracking and the prediction.
The smaller the mse of our Kalman-Filter compared to the standard implementation the greater the improvement.

\section{Data Acquisition}

We are going to acquire our data from billiard simulations and real videos.

\subsection{Simulation}

We construct a simulation to fully control every parameter and the state of the billiard game.
This helps us to accomplish a better conclusion to the performance in real video.
Our examination units are the framerate, the ball velocity, the sensor noise.

\paragraph{Framerate}

The framerate describes the framerate of the video, which is equal to the sampling rate of the Kalman-Filter.
A standard Kalman-Filter performs much worse on low framerate because it gets less data to process.
We test different framerate to inspect the performance of our implementation for different environments.
For realtime applications on mobile devices our implementation has to perform well on low framerates.
We are going to test the standard framerates 30 and 60 FPS and a very low framerate at 10 FPS.

\paragraph{Ball Velocity}

The ball velocity describes the speed of the ball, which depends on how hard the player hits the ball.
The velocity is measured in pixel per frame.
The get better analyze edge cases..

\section{Data Evaluation}

- mse

\bibliography{researchDesign} 
\bibliographystyle{abbrvnat}

\end{document}